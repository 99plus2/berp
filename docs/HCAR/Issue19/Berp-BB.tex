% Berp-BB.tex
\begin{hcarentry}[new]{Berp}
\label{berp}
\report{Bernie Pope}%11/10
\status{under development}
\makeheader

Berp is an implementation of Python 3. At its heart it is a translator
which takes Python code as input and generates Haskell code as output.
The Haskell code is fed into a Haskell compiler (GHC) for compilation
to machine code or interpretation as byte code. One of the main advantages
of this approach is that berp is able to use the rich
functionality provided by the GHC runtime system with minimal
implementation effort.
Berp provides both a compiler and an interactive interpreter, and for the
most part it can be used in the same way as CPython (the main Python
implementation). Although berp is in the early stages of development,
it is able to demonstrate some novel capabilities (compared to CPython),
such as tail-call optimisation and call-with-current-continuation.

The syntactic analysis component of berp is provided by a separate Haskell
library called language-python~\cref{language-python}, which can be used independently of berp to
produce tools for processing Python source.

Berp underwent a flurry of development activity in the first part of 2010,
but since then the pace slowed down as I worked on other projects.
Those other projects are now maturing, and I plan to return
to berp development soon. Berp is still missing support
for some critical features, such as module imports, and I hope to
remedy most of the major omissions by the end of the year.

\FurtherReading
\begin{compactitem}
\item \url{http://hackage.haskell.org/package/berp}
\item \url{http://github.com/bjpop/berp}
\item \url{http://github.com/bjpop/berp/wiki}
\end{compactitem}
\end{hcarentry}
